% Base de archivo LaTeX-article con paquetes necesarios cargados
% Jorge Eiras 2014 CC 

\usepackage[english]{babel}           % Permite escribir en Castellano (tildes y ñ) e implementa los contenidos en castellano.
\usepackage[utf8]{inputenc}           % Codificación de caracteres. WINDOWS Y LINUX.
%\usepackage[latin1]{inputenc}        % Codificacion de caracteres. MAC

%ALGUNOS PAQUETES INTERESANTES. ESTOS PAQUETES (LIBRERÍAS) YA ESTÁN EN NUESTRO EQUIPO. DESDE AQUÍ LOS ESTAMOS LLAMANDO.
\usepackage[T1]{fontenc} % Permite cambiar la fuente por defecto.
\usepackage{graphicx}    % Permite implementar imágenes.
\usepackage{color}       % Permite el uso de colores.
\usepackage{anysize}     % Permite modificar el tamaño de los márgenes.
\usepackage{multicol}    % Permite escribir a doble, triple...columna.
\usepackage{bm}          % Permite letras griegas en negrita (uso: \bm{\alpha}).
\usepackage{textcomp}    % Incluye símbolos específicos, pueden consultarse en la red.
\usepackage{eurosym}     % Permite escribir el símbolo € mediante la orden \euro.
\usepackage{amsthm}      % Paquete de la AMS para escribir teoremas.
\usepackage{amsmath,amsfonts} %Paquetes específicos de símbolos creados por la sociedad americana de matemáticas.
\usepackage{lineno}      % Permite numerar las líneas. Muy útil para borradores de trabajo en equipo.

%ORDENES PARA PERSONALIZAR EL FORMATO DE TU DOCUMENTO
\marginsize{1.5cm}{1cm}{1.5cm}{1.5cm} % MÁRGENES: Izq, Der, Sup, Inf.
\parindent=0mm                        % Sangría por defecto. 
\parskip=3mm                          % Espacio entre párrafos por defecto.
\renewcommand{\baselinestretch}{1}    % Interlineado.

%Commands for the title
\newcommand{\HRuleThin}{\rule{\linewidth}{0.5mm}}
\newcommand{\HRuleHuge}{\rule{\linewidth}{0.9mm}}

% To manage the url
\usepackage{url}

% To add the hyperlink to the index
\usepackage{hyperref}
\hypersetup{
	colorlinks,
	citecolor=black,
	filecolor=black,
	linkcolor=black,
	urlcolor=blue
}

% for the appendix
\usepackage[toc,page]{appendix}
\renewcommand\appendixtocname{Appendix}
\renewcommand{\appendixpagename}{Appendix}

%Per avere il conto delle pagine rispetto all'ultima pagina
\usepackage{fancyhdr}
\usepackage{lastpage}
\pagestyle{fancy} %Serve per il tipo della pagina per impostare le stile di header e footer
\fancyfoot{} %Rimuove la numerazione di default dello stile fancy
\fancyhead[R]{} %rimuovere la parte a destra dell'intestazione

\setlength{\headheight}{0.5cm} 
% No indentazione paragrafo
\setlength{\parindent}{0pt}

\usepackage{caption}
\usepackage{subcaption}